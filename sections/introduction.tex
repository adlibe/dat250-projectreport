\section{Introduction}
\label{sec:introduction}

The FeedApp project was developed as part of the DAT250 Advanced Software Technologies course to explore innovative solutions for managing polling systems effectively. The primary motivation was to create a scalable, secure, and user-friendly application for conducting polls while integrating advanced software technologies and best practices. This project addresses challenges such as ensuring reliable voting mechanisms, securing user data, and providing real-time analytics for decision-making.

The FeedApp allows users to create polls, vote, and view results through an intuitive web interface. Its design incorporates modern development practices such as RESTful APIs, microservices architecture, and CI/CD pipelines to ensure reliability and maintainability. By leveraging a relational database for core application data and a NoSQL database for analytics, the system is optimized to handle both structured and unstructured data efficiently. Furthermore, the inclusion of messaging systems for asynchronous communication enables real-time processing and storage of aggregated voting data.

A unique aspect of the FeedApp is its front-end implementation using the Elm programming language. Elm’s functional programming paradigm ensures a type-safe and error-free development environment, making it ideal for building high-reliability web applications. The project also integrates widely-used technologies like Java/SpringBoot, RabbitMQ, and MongoDB, chosen for their reliability, scalability, and compatibility.

The remainder of this report is structured as follows: First, we detail the FeedApp’s design and use cases, followed by an in-depth discussion of its architecture and technology stack. Next, we examine the prototype implementation and evaluate the project’s outcomes. Finally, we reflect on lessons learned and evaluate the effectiveness of the chosen technologies.

Through this report, we aim to present a comprehensive overview of the FeedApp, highlighting the challenges faced during its development and the solutions that emerged. The insights gained from this project demonstrate the potential of combining modern technologies to solve real-world problems in software development.

